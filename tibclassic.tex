\input tibex
\input pixidxfile
%排版藏文傳統長函書的包
\parindent=0pt%
%
\def\orot@te#1{\setbox0=\hbox{#1}%
  \special{x:gsave}\special{x:rotate 270}%
  \box0 \special{x:grestore}}
%
\NoInterLetterSkip
%
\newdimen\dimn%計算hoffset要用的臨時變量
\def\setpaperhalfa#1{%設置紙張大小,半A4/三分之一A3
  \ifnum#1=4 % 1/2 A4
     \pdfpageheight90mm\pdfpagewidth297mm%
     \vsize=55mm\hsize=243mm%
     \voffset=-10mm
  \else%
     \ifnum#1=3 % 1/3 A3
       \pdfpageheight99mm\pdfpagewidth420mm%
       \vsize=66mm\hsize=366mm%
       \voffset=-10.5mm%
     \fi\fi%
  \dimn=\pdfpagewidth \advance\dimn by -\hsize%
  \advance\dimn by -28mm \divide\dimn by 2 %
  \advance\dimn by -11mm %
  \hoffset=\dimn %
}%
\setpaperhalfa4%缺省的將紙張設爲半A4
%
\maxdepth=0pt%
\nopagenumbers%
% 這些與字體相關,應該定義一個宏讓用戶自己設缺省行高
\baselineskip=10.5mm plus1mm%
\lineskip=0mm plus1mm%
\lineskiplimit=0.2mm%
%
\parskip=0pt \parfillskip=0pt%
%
\newdimen\pb@xwidth \newdimen\pb@xheight \newdimen\ruleht@%
\newdimen\upm@rgin \upm@rgin=0pt%上邊框與文字的邊距
\newdimen\tmp@dim %
%\setboxupmargin設置上文字邊距
\def\setboxupmargin#1{\upm@rgin=#1}
% 變量初始化宏,在\output裡調用
\def\setboxsize{%
  \pb@xwidth=\hsize \pb@xheight=\vsize \ruleht@=1pt %
  \advance\pb@xwidth by 18pt \advance\pb@xwidth by 4.6mm %
  \advance\pb@xheight by 12pt \advance\pb@xheight by \upm@rgin \advance\pb@xheight by \upm@rgin %
  \advance\pb@xheight by \maxdepth }%
\def\d@rawpbox{%用ruleht@中所指定的線寬畫pboxwidth*pb@xheight方框
  \hrule height\ruleht@ width\pb@xwidth %上横边
  \hbox{%
    \vrule width\ruleht@ height\pb@xheight%
    \kern-\ruleht@ \kern\pb@xwidth \kern-\ruleht@%
    \vrule width\ruleht@ height\pb@xheight}%
  \hrule height\ruleht@ width\pb@xwidth %下横边
  }%
\def\shrinkpb@x{%将长宽减去2倍的ruleht@
  \advance\pb@xwidth by-\ruleht@ \advance\pb@xwidth by-\ruleht@ %
  \advance\pb@xheight by-\ruleht@ \advance\pb@xheight by-\ruleht@}%
\def\dblvrule#1{%雙竪線間隔2pt
  \vrule width1pt height#1%
  \kern2pt%
  \vrule width1pt height#1}%
\def\pdblvlin@{
  \dblvrule\pb@xheight}
% 由使用者设置这个变量告诉TibeX在最后一葉的正面加上་བྱོན
\newcount\totalpages \totalpages=0%
%
\def\putp@geno{\c@m=\pageno \advance\c@m by1 \divide\c@m by2 \tibpageno\c@m}
\def\tibpageno{\tibnumber} %設置頁碼風格,這裏是傳統的གཅིག་གཉིས་གསུམ།風格頁碼
\def\leftpagehd@r{%奇数页左边页眉框中的文字
    \tmp@dim=\pb@xheight \advance\tmp@dim by -12pt %
    \vbox to \tmp@dim{\hbox to 0pt{\orot@te{\hbox to \tmp@dim{ %左頁眉
        \kern-8pt\pre_phdr \hfil%
        \putp@geno %輸出頁碼
        \c@m=\totalpages %
        \ifnum\pageno>\c@m \relax%
        \else%
            \advance\c@m by-\pageno%
            \ifnum\c@m<2 ་བྱོན། \fi\fi %加在最後一頁
        \hfil\post_phdr\kern12pt}}%
    \hss}\vss} %左頁眉
  }%
% 控制是否加邊框及頁眉
\newif\ifframe \framefalse % 是否加最外邊框
\newif\ifinnerframe \innerframetrue % 是否不加内框
\def\noinnerframe{ % 不加内框,祇加上左頁眉並留空
    \innerframefalse %
    }
\newif\ifmultiframe \multiframefalse % 臨時變量,是否是奇數頁加三内框
%
\def\setppix#1{\XeTeXpicfile#1 height \pb@xheight}
\def\pboxlpic{} %設置這兩個宏來讓\output輸出圖片
\def\pboxrpic{} %
%中間的正文框,\ifpixbox則給兩邊加上圖片框
\newdimen\pixb@xwidth
\def\pb@xpic#1{
  \vbox to \pb@xheight{
    \pixb@xwidth=\pb@xheight %
    \multiply\pixb@xwidth by3 \divide\pixb@xwidth by4% 寬=3/4高
    \hbox to \pixb@xwidth{#1\hss}}}%
\def\midtextb@x{ %
  \ifpixbox %
    %\message{ pixpageno=\the\pixp@geno. } %
    \ifnum\pageno=\pixp@geno %
      \hrule height\ruleht@ width\pb@xwidth %上横边
      \hbox to \pb@xwidth{%
        \vrule width\ruleht@ height\pb@xheight%左豎邊
        %\kern-\ruleht@ \kern\pb@xwidth \kern-\ruleht@%
        %\show\pboxlpic %
        \pb@xpic{\pboxlpic} %左圖片
        \pdblvlin@ %雙竪綫
        %\kern-\pixb@xwidth \kern-pixb@xwidth \kern-8pt%中空
        \hfil %
        \pdblvlin@ %雙竪綫
        \pb@xpic{\pboxrpic} %右圖片
        \vrule width\ruleht@ height\pb@xheight} %右豎邊
      \hrule height\ruleht@ width\pb@xwidth %下横边
    \else % 這一頁沒圖片
      %\message{===no. \the\pageno{\the\pixp@geno} is not pix page, .} %
      \d@rawpbox \fi % \ifnum
  \else % not pixbox
    %\message{===no pixbox.} %
    \d@rawpbox %
  \fi} %\ifpixbox
\def\estim@tepixboxwidth{ %預估\pixboxwidth的值,在setparshape中要用到
  \pixb@xwidth=\vsize %
  \advance\pixb@xwidth by 27pt %
  \advance\pixb@xwidth by-6pt \advance\pixb@xwidth by\maxdepth %\shrinkbox,+2pt之後的内框高
  \multiply\pixb@xwidth by 3 \divide\pixb@xwidth by 4} % 3/4
%
% Output routine
%
\def\onepageout#1{%
 \shipout\vbox{%方框
  %\message{ BG[pixpageno=\the\pixp@geno.] }
  \ifframe%
    \setboxsize %初始化計算方框大小的寄存器
    \offinterlineskip %
    \vskip-4pt \vskip-\upm@rgin %
    \hbox{\kern-16pt\vbox to 4pt{ %粗外框
       \ruleht@=2pt \d@rawpbox %
       \vss}}%
    \ruleht@=4pt \shrinkpb@x \advance\pb@xheight by2pt%
    \ifinnerframe % 需要加内框
        \ifodd\pageno \multiframetrue %
        \else\ifnum\pageno=2 \multiframetrue %特例,用於P2不想用四重方框時
        \else\multiframefalse\fi\fi %
        %
        \ifmultiframe % 奇數頁三内框
            \ruleht@=1pt %
            \hbox{ % 内含三个框
                \kern-12.8pt %
                \vbox to 2pt{% 左小框
                    \hrule width36pt height\ruleht@%
                    \hbox to 36pt{%
                        \vrule width\ruleht@ height\pb@xheight%
                        \hfil\kern15pt\leftpagehd@r\hfil % 左页眉文字
                        \vrule width\ruleht@ height\pb@xheight%
                    }%
                    \hrule width36pt height\ruleht@ \vss} %
                \advance\pb@xwidth by-36pt %
                \advance\pb@xwidth by-36pt % 算中框的寬度
                \kern1pt%
                \vbox to 2pt{% 正文框
                    \advance\pb@xwidth by -3.5pt \midtextb@x \vss} %
                \kern1.25pt %
                \vbox to 2pt{% 右小框
                    \pb@xwidth=36pt \d@rawpbox \vss} %
            } %\hbox
        \else % 偶數頁套细方框
            \hbox{\kern-12pt\vbox to 2pt{
               \ruleht@=1pt \d@rawpbox %
               \vss}} \fi %\ifmultiframe
    \else %不需要加内框
        \ifodd\pageno % 奇數頁P5等加頁眉
            \hbox{ % 内含三个框
                \kern-12.8pt %
                \vbox to 2pt{% 左小框
                    \hbox to 36pt{%
                        \hfil\kern10pt\leftpagehd@r\hfil % 左页眉文字
                    }\vss}} \fi %\ifodd
    \fi %\ifinnerframe 
  \fi %\ifframe
  \vskip2pt \vskip\upm@rgin %
  \vbox to \vsize{%页面内容
      #1 \boxmaxdepth=\maxdepth }%
  }%
  \nextp@gepixs % 讀入下一頁圖片索引
  %\message{ [pixpageno=\the\pixp@geno.] }
  %\showpixvars %
  \advancepageno %
}%
%设置为输出宏
\output{\onepageout{\unvbox255}}
%
% 用於P1,P2,P3的四層方框
%
\newif\ifpagehdr\pagehdrfalse %是否在四层方框中放左页眉
%\newdimen\b@xwidth %四层方框宽
%\newdimen\b@xheight %高
\newcount\quadboxgapstyle \quadboxgapstyle=1
\newif\ifpixbox\pixboxfalse %是否左右留圖片框 %普通四層框
\newdimen\d@r %用来传递dimension的临时寄存器
\newdimen\d@t %存放標題兩邊的長框寬度的臨時寄存器
\def\setboxwidth#1{\pb@xwidth=#1} %设置方框的宽度
\def\setboxheight#1{\pb@xheight=#1} %设置高度
\def\shrinkboxsize{%将宽、高分别减少两倍\d@r
  \advance\pb@xwidth by-\d@r \advance\pb@xwidth by-\d@r %
  \advance\pb@xheight by-\d@r \advance\pb@xheight by-\d@r}%
\def\setg@pvalue{%根據\ifpixbox來設置二三層框間距
    \ifpixbox % 帶圖片的四層框
        \d@r=1.6pc \d@t=\d@r%
    \else % 普通四層框
        \ifnum\quadboxgapstyle=1 %風格1,二、三框間上下空1/5
            \d@r=\pb@xheight \divide\d@r by 5 \d@t=\d@r %
            \advance\d@r by1pt %
        \else\ifnum\quadboxgapstyle=2 
            \d@r=2.4pc \d@t=\d@r% 二三層間小間距
        \fi\fi\fi}%
\def\dblvlin@{%
  \dblvrule\pb@xheight}%
\def\dr@wbox{%用d@r中所指定的線寬畫方框
  \hrule height\d@r width\pb@xwidth %上横边1
  \hbox{%
    \vrule width\d@r height\pb@xheight%
    \kern-\d@r \kern\pb@xwidth \kern-\d@r%
    \vrule width\d@r height\pb@xheight}%
  \hrule height\d@r width\pb@xwidth %下横边1
  }%
%
\def\hdr_up{} %調節頁眉位置,用\kern控制頁眉上下
\def\pre_phdr{} %頁碼前頁眉
\def\post_phdr{} %頁碼後頁眉
\newdimen\hdroffs@t \hdroffs@t=0pt
\def\putp@gehdr{%放頁眉,用於P2,P3
  \dimn=\pb@xheight \advance\dimn by-.25em %
  \hdroffs@t=-\dimn \divide\hdroffs@t by 2 %
  \vbox to 2cm{\vskip\hdroffs@t \hdr_up \hbox to 0pt{%
    \orot@te{\hbox to \dimn{%
      \pre_phdr\hfil%
      \putp@geno%
      \hfil\post_phdr}}\hss}\vss}}%
\def\lmultivrul@s{%左邊第四方框內的幾條竪線
  \d@t=\pb@xwidth \divide\d@t by 24 %
  \kern\d@t \kern1.2pc %
  \ifpagehdr\putp@gehdr\fi \kern-1.2pc\kern\d@t %1/12
  \dblvlin@ %4pt
  \multiply\d@t by2 \divide\d@t by5 \kern\d@t \dblvlin@ % 1/60+4pt
  }%
\def\rmultivrul@s{%右邊第四方框內的幾條竪線
  \d@t=\pb@xwidth \divide\d@t by 60 %
  \dblvlin@ \kern\d@t%
  \dblvlin@ \multiply\d@t by 5 \kern\d@t}%
\def\rem@inedwidth#1{%計算剩下的寬度之半,存在\d@r中
  \d@t=\pb@xwidth \divide\d@t by 10 % 1/12 + 1/60 = 1/10
  \d@r=\pb@xwidth \advance\d@r by-\d@t \advance\d@r by-\d@t %
  \advance\d@r by-18pt %
  \setbox0=#1 %
  \advance\d@r by-\wd0 %文字的寬度
  \advance\d@r by-.7pt % 調整對齊
  \divide\d@r by 2}%
\def\setpixboxwidth{ %計算圖片框寬度放在\dt中
  \d@t=\pb@xheight \multiply\d@t by3 \divide\d@t by4 }%
\def\rem@inedwidthpic{ %計算剩下的寬度存在d@r中
  \d@r=\pb@xwidth \advance\d@r by-2.8pc %左第一框
  \advance\d@r by -4pt %雙竪綫
  \setpixboxwidth \multiply\d@t by2 %兩個圖片框
  \advance\d@r by -\d@t %
  \advance\d@r by -4pt \advance\d@r by -4pt %左右各有一個雙竪綫
  \advance\d@r by -2.7pt}%
%
\def\quadrabox#1{%四層方框的主宏
\offinterlineskip%
\hbox{\vbox to 4pt{%第一方框
  \d@r=2pt\dr@wbox%
  \vss}}%
\d@r=4pt\shrinkboxsize%
\advance\pb@xheight by 2pt%
\hbox{\kern4pt\vbox to 1pt{%第二方框
  \d@r=1pt\dr@wbox%

  \vss}}%
\setg@pvalue %設置二三層框之間的間距
\shrinkboxsize %
\vskip\d@t %暫存,第四方框時橫向留空時要用
\hbox{\kern5pt\kern\d@t%
  \vbox to 3pt{%第三方框
    \d@r=1pt\dr@wbox%
    \vss}}%
\d@r=3pt\shrinkboxsize%
\hbox{\kern8pt\kern\d@t\vbox{%第四方框
   \hrule height1pt width\pb@xwidth %上横边4
   \hbox to\pb@xwidth{%
     \vrule width1pt height\pb@xheight % 1pt
     \hbox{%
       \ifpixbox %左右留圖片框
           \hskip1.8pc \ifpagehdr\putp@gehdr\fi \hskip1pc %頁眉
           \dblvlin@ %左第一框帶頁眉
           \setpixboxwidth %計算圖片框寬度
           \vbox to \pb@xheight{\hbox to \d@t{\pboxlpic\hss}} %
           %\kern-.7pt
           \dblvlin@ %左圖片框
           \rem@inedwidthpic \kern\d@r \kern-1.4pt%中間文字留空
           \dblvlin@ \setpixboxwidth %右圖片框
           \vbox to \pb@xheight{\hbox to \d@t{\pboxrpic\hss}} %\kern-.7pt %
       \else %普通四層框
           \lmultivrul@s %
           %\tracingcommands=3 \tracingmacros=3
           \rem@inedwidth{#1} %
           \hskip\d@r \hfil %
           \vbox{\vfil \box0 \vskip7pt\vfil}% 
           \hfil \hskip\d@r %
           \rmultivrul@s %
       \fi}%
     \vrule width1pt height\pb@xheight\hss}%
   \hrule height1pt width\pb@xwidth %下横边4
   }}%
}%
%
\newcount\t@mpmod
\def\modulo#1#2{ % 求余数
    \t@mpmod=#1 %
    \divide\t@mpmod by#2 \multiply\t@mpmod by#2 \multiply\t@mpmod by-1 %
    \advance\t@mpmod by #1}
%用于设置从P5开始的正文段落形状的宏
\newdimen\@ddlskip \@ddlskip=0pt
\newdimen\@venlskip \@venlskip=0pt
\newdimen\@ddtexthsize % 用來計算奇數頁文字寬度
\newdimen\ev@ntexthsize % 用來計算偶數頁文字寬度
\def\slin@il{\@ddlskip\@ddtexthsize}% short line indent length
\def\twoslin@il{\slin@il\slin@il}%兩個短行
\def\threeslin@il{\twoslin@il\slin@il}%
\def\fourslin@il{\twoslin@il\twoslin@il}%
\def\fiveslin@il{\threeslin@il\twoslin@il}%
\def\sixslin@il{\threeslin@il\threeslin@il}%
\def\sevenslin@il{\fourslin@il\threeslin@il}%最多一頁七個短行
\def\llin@il{\@venlskip\ev@ntexthsize}%
\def\twollin@il{\llin@il\llin@il}%
\def\threellin@il{\twollin@il\llin@il}%
\def\fourllin@il{\twollin@il\twollin@il}%
\def\fivellin@il{\threellin@il\twollin@il}%
\def\sixllin@il{\threellin@il\threellin@il}%
\def\sevenllin@il{\fourllin@il\threellin@il}%最多一頁七個長行
\def\pfiv@{\fiveslin@il \fivellin@il}%
\def\fourpfiv@{\pfiv@\pfiv@\pfiv@\pfiv@}%
\def\sixteenpfiv@{\fourpfiv@\fourpfiv@\fourpfiv@\fourpfiv@}%
\def\thirtytwopfiv@{\sixteenpfiv@\sixteenpfiv@}%
\def\p@six{\sixslin@il \sixllin@il}%
\def\f@urpsix{\p@six\p@six\p@six\p@six}%
\def\sixte@npsix{\f@urpsix\f@urpsix\f@urpsix\f@urpsix}%
\def\thirtytw@psix{\sixte@npsix\sixte@npsix}%
\def\psev@n{\sevenslin@il\sevenllin@il}
\def\fourpsev@n{\psev@n\psev@n\psev@n\psev@n}%
\def\sixteenpsev@n{\fourpsev@n\fourpsev@n\fourpsev@n\fourpsev@n}%
\def\thirtytwopsev@n{\sixteenpsev@n\sixteenpsev@n}%
\def\setparshape#1#2{%{#1=5/6/7}{#2=该段文字大致最多的页数(<32*2)}
    %设置该段文字为5/6/7個短行,接5/6/7個長行
    \prevlinesrem@ined=0
    \@ddtexthsize=\hsize %
    \ev@ntexthsize=\hsize \advance\ev@ntexthsize by 4pt %
    \ifinnerframe % 帶内框,外框已經左移了12pt(0.42cm)
        \advance\@ddtexthsize by-2.4cm % 帶内框時P5左右各空1.2cm(36pt)
        \@ddlskip=35pt \@venlskip=-2pt % P6左右皆不留空
        \ifpixbox % 内框左右有圖片
            \estim@tepixboxwidth % 預估圖片框寬度,設置\pixb@xwidth
            \advance\@ddtexthsize by -\pixb@xwidth %
            \advance\@ddtexthsize by -\pixb@xwidth %
            \advance\@ddtexthsize by -4mm %
            \advance\@ddlskip by \pixb@xwidth %
            \advance\@ddlskip by 2mm %
            %\message{"oddlskip=\the\@ddlskip."}
        \fi
    \else % 不帶内框,排版完成後再嵌入僅含有左右黑條的P45.pdf
        \advance\@ddtexthsize by -21.47mm % 不帶内框時P5左19.95mm右9mm
        %\showthe\@ddtexthsize %
        \@ddlskip=45.3pt \@venlskip=16.6pt % P5 19.95mm(57pt) / P6 0.9cm(26pt)
        \advance\ev@ntexthsize by-12.76mm % P6左右黑邊合計
        %\showthe\ev@ntexthsize %
    \fi % \ifinnerframe
    \ifnum#1=5 % 設置n頁的段落形狀
        \ifnum#2<5 \parshape=40 \fourpfiv@ %4*2頁
        \else\ifnum#2<17 \parshape=160 \sixteenpfiv@ %16*2頁
        \else \parshape=320 \thirtytwopfiv@ \fi\fi %最多不超過32*2頁
    \else\ifnum#1=6 %
        \ifnum#2<5 \parshape=48 \f@urpsix %
        \else\ifnum#2<17 \parshape=192 \sixte@npsix %
        \else \parshape=384 \thirtytw@psix \fi\fi %
    \else\ifnum#1=7 %
        \ifnum#2<5 \parshape=56 \fourpsev@n %
        \else\ifnum#2<17 \parshape=224 \sixteenpsev@n %
        \else \parshape=448 \thirtytwopsev@n \fi\fi %
    \fi\fi\fi} %
\def\setrestedparsh@pe#1#2{ %{5/6/7}{一個正反面中還剩下的行數}祇考虑一段2*16頁
    %\message{arg1=#1, arg2=#2}
    \ifnum#1=5 %
        \ifcase#2 \parshape=160 \sixteenpsev@n \relax %
        \or \parshape=161 \llin@il \sixteenpfiv@ %1行
        \or \parshape=162 \twollin@il \sixteenpfiv@ %2行
        \or \parshape=163 \threellin@il \sixteenpfiv@ %3行
        \or \parshape=164 \fourllin@il \sixteenpfiv@ %4行
        \or \parshape=165 \fivellin@il \sixteenpfiv@ %5行
        \or \parshape=166 \slin@il\fivellin@il \sixteenpfiv@ %1+5行
        \or \parshape=167 \twoslin@il\fivellin@il \sixteenpfiv@ %2+5行
        \or \parshape=168 \threeslin@il\fivellin@il \sixteenpfiv@ %3+5行
        \or \parshape=169 \fourslin@il\fivellin@il \sixteenpfiv@ %4+5行
        \or \parshape=160 \sixteenpfiv@ \fi %5+5
    \else\ifnum#1=6 %
        \ifcase#2 \parshape=192 \sixte@npsix \relax %
        \or \parshape=193 \llin@il \sixte@npsix %1
        \or \parshape=194 \twollin@il \sixte@npsix %2
        \or \parshape=195 \threellin@il \sixte@npsix %3
        \or \parshape=196 \fourllin@il \sixte@npsix %4
        \or \parshape=197 \fivellin@il \sixte@npsix %5
        \or \parshape=198 \sixllin@il \sixte@npsix %6
        \or \parshape=199 \slin@il\sixllin@il \sixte@npsix %1+6
        \or \parshape=200 \twoslin@il\sixllin@il \sixte@npsix %2+6
        \or \parshape=201 \threeslin@il\sixllin@il \sixte@npsix %3+6
        \or \parshape=202 \fourslin@il\sixllin@il \sixte@npsix %4+6
        \or \parshape=203 \fiveslin@il\sixllin@il \sixte@npsix %5+6
        \or \parshape=192 \sixte@npsix \fi %6+6 
    \else\ifnum#1=7 %
        \ifcase#2 \parshape=224 \sixteenpsev@n \relax %
        \or \parshape=225 \llin@il \sixteenpsev@n %前段剩1行
        \or \parshape=226 \twollin@il \sixteenpsev@n %
        \or \parshape=227 \threellin@il \sixteenpsev@n %
        \or \parshape=228 \fourllin@il \sixteenpsev@n %
        \or \parshape=229 \fivellin@il \sixteenpsev@n %
        \or \parshape=230 \sixllin@il \sixteenpsev@n %
        \or \parshape=231 \sevenllin@il \sixteenpsev@n %
        \or \parshape=232 \slin@il\sevenllin@il \sixteenpsev@n %1+7
        \or \parshape=233 \twoslin@il\sevenllin@il \sixteenpsev@n %
        \or \parshape=234 \threeslin@il\sevenllin@il \sixteenpsev@n %
        \or \parshape=235 \fourslin@il\sevenllin@il \sixteenpsev@n %
        \or \parshape=236 \fiveslin@il\sevenllin@il \sixteenpsev@n %
        \or \parshape=237 \sixslin@il\sevenllin@il \sixteenpsev@n %
        \or \parshape=224 \sixteenpsev@n \relax \fi %
    \fi\fi\fi} %
\newcount\prevlinesrem@ined \prevlinesrem@ined=0 %上一段所剩最後一頁中行
\newcount\t@i \newcount\t@j % 臨時變量
\def\setnewparshape#1{ %{5/6/7}根據上一段剩餘的行數設置新一段的parshape
    %\message@!\the\prevlinesrem@ined}
    \t@j=#1 \multiply\t@j by 2 %
    \t@i=\prevgraf \advance\t@i by \prevlinesrem@ined %加上上一段剩下的行數
    %\showthe\t@i \showthe\t@j %
    \modulo{\the\t@i}{\the\t@j} %計算本段開始時已排滿的行數
    \prevlinesrem@ined=\t@mpmod %更新\prevlinesrem@ined下一段要用
    \advance\t@j by -\t@mpmod % 一葉中總行數減去已排滿的行數
    %\showthe\t@j \s@owthe\prevlinesrem@ined %
    \setrestedparsh@pe{#1}{\the\t@j}}
%簡單的單層標題方框
\def\titlebox#1{\hbox{\vbox{
\hrule height1pt\hbox{\vrule width1pt\kern4pc\vrule width1pt
\vbox{\kern9pt#1\kern9pt}\vrule width1pt\kern4pc\vrule width1pt}\hrule height1pt}
}}
%設置自動加在奇數頁首的雲頭符
\newbox\pghdr%
\TibHeaderBox=\pghdr%
\def\header{{\SamChe ༄༅\kern.01pt།\hskip2.7em plus .5em minus.5em།}}%
\setbox\pghdr=\hbox{\header}%
\Sam
