\input tibex
%排版藏文傳統長函書的包
\parindent=0pt%
%
\def\orotate#1{\setbox0=\hbox{#1}%
  \special{x:gsave}\special{x:rotate 270}%
  \box0 \special{x:grestore}}
%
\NoInterLetterSkip
%
\newdimen\dimn%計算hoffset要用的臨時變量
\def\setpaperhalfa#1{%設置紙張大小,半A4/三分之一A3
  \ifnum#1=4 % 1/2 A4
     \pdfpageheight90mm\pdfpagewidth297mm%
     \vsize=55mm\hsize=243mm%
     \voffset=-10mm
  \else%
     \ifnum#1=3 % 1/3 A3
       \pdfpageheight99mm\pdfpagewidth420mm%
       \vsize=66mm\hsize=366mm%
       \voffset=-8mm%
     \fi\fi%
  \dimn=\pdfpagewidth \advance\dimn by -\hsize%
  \advance\dimn by -28mm \divide\dimn by 2 %
  \advance\dimn by -11mm %
  \hoffset=\dimn %
}%
\setpaperhalfa4%缺省的將紙張設爲半A4
%
\maxdepth=0pt%
\nopagenumbers%
% 這些與字體相關,應該定義一個宏讓用戶自己設缺省行高
\baselineskip=10.5mm plus1mm%
\lineskip=0mm plus1mm%
\lineskiplimit=0.2mm%
%
\parskip=0pt \parfillskip=0pt%
%
\newdimen\pboxwidth \newdimen\pboxheight \newdimen\ruleht%
\newdimen\tmpdim %
% 變量初始化宏,在\output裡調用
\def\setboxsize{%
  \pboxwidth=\hsize \pboxheight=\vsize \ruleht=1pt%
  \advance\pboxwidth by 18pt \advance\pboxwidth by 4.6mm%
  \advance\pboxheight by 27pt \advance\pboxheight by \maxdepth}%
\def\drawpbox{%用ruleht中所指定的線寬畫pboxwidth*pboxheight方框
  \hrule height\ruleht width\pboxwidth %上横边1
  \hbox{%
    \vrule width\ruleht height\pboxheight%
    \kern-\ruleht \kern\pboxwidth \kern-\ruleht%
    \vrule width\ruleht height\pboxheight}%
  \hrule height\ruleht width\pboxwidth %下横边1
  }%
\def\shrinkpbox{%将长宽减去2倍的ruleht
  \advance\pboxwidth by-\ruleht \advance\pboxwidth by-\ruleht %
  \advance\pboxheight by-\ruleht \advance\pboxheight by-\ruleht}%
% 由使用者设置这个变量告诉TibeX在最后一页的正面加上་བྱོན
\newcount\totalpages \totalpages=0%
%
\def\leftpagehdr{%奇数页左边页眉框中的文字
    \tmpdim=\pboxheight \advance\tmpdim by -12pt %
    \vbox to \tmpdim{\hbox to 0pt{\orotate{\hbox to \tmpdim{ %左頁眉
        \kern-8pt\pre_phdr \hfil%
        \m=\pageno \advance\m by 1 \divide\m by 2%
        \tibnumber\m%
        \m=\totalpages %
        \ifnum\pageno>\m \relax%
        \else%
            \advance\m by-\pageno%
            \ifnum\m<2 ་བྱོན། \fi\fi %加在最後一頁
        \hfil\post_phdr\kern12pt}}%
    \hss}\vss} %左頁眉
  }%
% 控制是否加邊框及頁眉
\newif\ifframe \framefalse % 是否加最外邊框
\newif\ifinnerframe \innerframetrue % 是否不加内框
\def\noinnerframe{ % 不加内框,祇加上左頁眉並留空
    \innerframefalse % 尚需修改setparshape,將其中硬編碼的數改爲用變量的
    }
\newif\ifmultiframe \multiframefalse % 臨時變量,是否是奇數頁加三内框
\def\onepageout#1{%
 \shipout\vbox{%方框
  \ifframe%
    \setboxsize %初始化計算方框大小的寄存器
    \offinterlineskip %
    \vskip-10pt%
    \hbox{\kern-16pt\vbox to 4pt{ %粗外框
       \ruleht=2pt \drawpbox %
       \vss}}%
    \ruleht=4pt \shrinkpbox \advance\pboxheight by2pt%
    \ifinnerframe % 需要加内框
        \ifodd\pageno \multiframetrue %
        \else\ifnum\pageno=2 \multiframetrue %特例,用於P2不想用四重方框時
        \else\multiframefalse\fi\fi %
        %
        \ifmultiframe % 奇數頁三内框
            \ruleht=1pt %
            \hbox{ % 内含三个框
                \kern-12.8pt %
                \vbox to 2pt{% 左小框
                    \hrule width36pt height\ruleht%
                    \hbox to 36pt{%
                        \vrule width\ruleht height\pboxheight%
                        \hfil\kern10pt\leftpagehdr\hfil % 左页眉文字
                        \vrule width\ruleht height\pboxheight%
                    }%
                    \hrule width36pt height\ruleht \vss} %
                \advance\pboxwidth by-36pt %
                \advance\pboxwidth by-36pt % 算中框的寬度
                \kern1pt%
                \vbox to 2pt{% 正文框
                    \advance\pboxwidth by -3.5pt \drawpbox \vss} %
                \kern1.25pt %
                \vbox to 2pt{% 右小框
                    \pboxwidth=36pt \drawpbox \vss} %
            } %\hbox
        \else % 偶數頁套细方框
            \hbox{\kern-12pt\vbox to 2pt{
               \ruleht=1pt \drawpbox %
               \vss}} \fi %\ifmultiframe
    \else %不需要加内框
        \ifodd\pageno % 奇數頁P5等加頁眉
            \hbox{ % 内含三个框
                \kern-12.8pt %
                \vbox to 2pt{% 左小框
                    \hbox to 36pt{%
                        \hfil\kern10pt\leftpagehdr\hfil % 左页眉文字
                    }\vss}} \fi %\ifodd
    \fi %\ifinnerframe 
  \fi %\ifframe
  \vskip9pt
  \vbox to \vsize{%页面内容
      #1 \boxmaxdepth=\maxdepth }%
  }%
  \advancepageno %
}%
%设置为输出宏
\output{\onepageout{\unvbox255}}
%用於P1,P2,P3的四層方框
\newif\ifpagehdr\pagehdrfalse %是否在四层方框中放左页眉
\newdimen\boxwidth %四层方框宽
\newdimen\boxheight %高
\newdimen\dr %用来传递dimension的临时寄存器
\newdimen\dt %存放標題兩邊的長框寬度的臨時寄存器
\def\setboxwidth#1{\boxwidth=#1} %设置方框的宽度
\def\setboxheight#1{\boxheight=#1} %设置高度
\def\shrinkboxsize{%将宽、高分别减少两倍\dr
  \advance\boxwidth by-\dr \advance\boxwidth by-\dr %
  \advance\boxheight by-\dr \advance\boxheight by-\dr}%
\def\dblvrule{%雙竪線間隔2pt
  \vrule width1pt height\boxheight%
  \kern2pt%
  \vrule width1pt height\boxheight}%
\def\drawbox{%用dr中所指定的線寬畫方框
  \hrule height\dr width\boxwidth %上横边1
  \hbox{%
    \vrule width\dr height\boxheight%
    \kern-\dr \kern\boxwidth \kern-\dr%
    \vrule width\dr height\boxheight}%
  \hrule height\dr width\boxwidth %下横边1
  }%
%
\def\hdr_up{}%缺省定義為空
\def\pre_phdr{}%
\def\post_phdr{}%
\newdimen\hdroffset \hdroffset=0pt
\def\putpagehdr{%放頁眉,用於P2,P3
  \dimn=\boxheight \advance\dimn by-.25em %
  \hdroffset=-\dimn \divide\hdroffset by 2 %
  \vbox to 2cm{\vskip\hdroffset \hdr_up \hbox to 0pt{%
    \orotate{\hbox to \dimn{%
      \pre_phdr\hfil%
      \ifnum\pageno=2 གཅིག %
      \else\ifnum\pageno=3 གཉིས %
      \fi\fi%
      \hfil\post_phdr}}\hss}\vss}}%
\def\lmultivrules{%左邊第四方框內的幾條竪線
  \dt=\boxwidth \divide\dt by 24 %
  \kern\dt \kern1.2pc %
  \ifpagehdr\putpagehdr\fi \kern-1.2pc\kern\dt %1/12
  \dblvrule %4pt
  \multiply\dt by2 \divide\dt by5 \kern\dt \dblvrule % 1/60+4pt
  }%
\def\rmultivrules{%右邊第四方框內的幾條竪線
  \dt=\boxwidth \divide\dt by 60 %
  \dblvrule \kern\dt%
  \dblvrule \multiply\dt by 5 \kern\dt}%
\def\remainedwidth#1{%計算剩下的寬度存在dr中
  \dt=\boxwidth \divide\dt by 10 % 1/12 + 1/60 = 1/10
  \dr=\boxwidth \advance\dr by-\dt \advance\dr by-\dt %
  \advance\dr by-18pt %
  \setbox0=#1 %
  \advance\dr by-\wd0 %文字的寬度
  \advance\dr by-.7pt % 調整對齊
  \divide\dr by 2}%
%
\def\quadrabox#1{%四層方框的主宏
\offinterlineskip%
\hbox{\vbox to 4pt{%第一方框
  \dr=2pt\drawbox%
  \vss}}%
\dr=4pt\shrinkboxsize%
\advance\boxheight by 2pt%
\hbox{\kern4pt\vbox to 1pt{%第二方框
  \dr=1pt\drawbox%
  \vss}}%
\dr=\boxheight \divide\dr by 5 \dt=\dr%第三框上下各留1/5空
\advance\dr by1pt \shrinkboxsize%
\vskip\dt %暫存,第四方框時橫向留空時要用
\hbox{\kern5pt\kern\dt%
  \vbox to 3pt{%第三方框
    \dr=1pt\drawbox%
    \vss}}%
\dr=3pt\shrinkboxsize%
\hbox{\kern8pt\kern\dt\vbox{%第四方框
   \hrule height1pt width\boxwidth %上横边4
   \hbox to\boxwidth{%
     \vrule width1pt height\boxheight % 1pt
     \hbox{%
       \lmultivrules %
       %\tracingcommands=3 \tracingmacros=3
       \remainedwidth{#1} %
       \hskip\dr \hfil %
       \vbox{\vfil \box0 \vskip7pt\vfil}% 
       \hfil \hskip\dr %
       \rmultivrules}%
     \vrule width1pt height\boxheight\hss}%
   \hrule height1pt width\boxwidth %下横边4
   }}%
}%
%
\newcount\tmpmod
\def\modulo#1#2{ % 求余数
    \tmpmod=#1 %
    \divide\tmpmod by#2 \multiply\tmpmod by#2 \multiply\tmpmod by-1 %
    \advance\tmpmod by #1}
%用于设置从P5开始的正文段落形状的宏
\newdimen\oddlskip \oddlskip=0pt
\newdimen\evenlskip \evenlskip=0pt
\newdimen\odtexthsize % 用來計算奇數頁文字寬度
\newdimen\evtexthsize % 用來計算偶數頁文字寬度
\def\slineil{\oddlskip\odtexthsize}%
\def\twoslineil{\slineil\slineil}
\def\threeslineil{\twoslineil\slineil}
\def\fourslineil{\twoslineil\twoslineil}
\def\fiveslineil{\threeslineil\twoslineil}%
\def\sixslineil{\fiveslineil\slineil}%
\def\llineil{\evenlskip\evtexthsize}%
\def\twollineil{\llineil\llineil}
\def\threellineil{\twollineil\llineil}
\def\fourllineil{\twollineil\twollineil}
\def\fivellineil{\threellineil\twollineil}%
\def\sixllineil{\fivellineil\llineil}%
\def\pfive{\fiveslineil \fivellineil}%
\def\fourpfive{\pfive\pfive\pfive\pfive}%
\def\sixteenpfive{\fourpfive\fourpfive\fourpfive\fourpfive}%
\def\thirtytwopfive{\sixteenpfive\sixteenpfive\sixteenpfive\sixteenpfive}%
\def\psix{\sixslineil \sixllineil}%
\def\fourpsix{\psix\psix\psix\psix}%
\def\sixteenpsix{\fourpsix\fourpsix\fourpsix\fourpsix}%
\def\thirtytwopsix{\sixteenpsix\sixteenpsix\sixteenpsix\sixteenpsix}%
\def\setparshape#1#2{%{5/6}{该段文字大致最多的页数(<32*2)}
    %设置该段文字为5/6個短行,接5/6個長行
    \odtexthsize=\hsize %
    \evtexthsize=\hsize %
    \ifinnerframe % 帶内框,外框已經左移了12pt(0.42cm)
        \advance\odtexthsize by-2.4cm % 帶内框時P5左右各空1.2cm(36pt)
        \oddlskip=35pt \evenlskip=0pt % P6左右皆不留空
    \else % 不帶内框
        \advance\odtexthsize by -2.16cm % 不帶内框時P5左2.1cm右0.9cm
        %\showthe\odtexthsize %
        \oddlskip=48.3pt \evenlskip=14pt % P5 2.1cm(60pt) / P6 0.9cm(26pt)
        \advance\evtexthsize by-0.96cm % P6左右黑邊合計1.8cm
        %\showthe\evtexthsize %
    \fi % \ifinnerframe
    \ifnum#1=5 %
        \ifnum#2<4 \parshape=40 \fourpfive %
        \else\ifnum#2<16 \parshape=160 \sixteenpfive %
        \else \parshape=640 \thirtytwopfive \fi\fi %
    \else\ifnum#1=6 %
        \ifnum#2<4 \parshape=48 \fourpsix %
        \else\ifnum#2<16 \parshape=192 \sixteenpsix %
        \else \parshape=768 \thirtytwopsix \fi\fi\fi\fi} %
\def\setrestedparshape#1#2{ %{5/6}{一個正反面中還剩下的行數}一段不超過2*16頁
    \ifnum#1=5 %
        \ifcase#2 \relax %
        \or \parshape=161 \llineil \sixteenpfive %1行
        \or \parshape=162 \twollineil \sixteenpfive %2行
        \or \parshape=163 \threellineil \sixteenpfive %3行
        \or \parshape=164 \fourllineil \sixteenpfive %4行
        \or \parshape=165 \fivellineil \sixteenpfive %5行
        \or \parshape=166 \slineil\fivellineil \sixteenpfive %1+5行
        \or \parshape=167 \twoslineil\fivellineil \sixteenpfive %2+5行
        \or \parshape=168 \threeslineil\fivellineil \sixteenpfive %3+5行
        \or \parshape=169 \fourslineil\fivellineil \sixteenpfive \fi %4+5行
    \else\ifnum#1=6 %
        \ifcase#2 \relax %
        \or \parshape=193 \llineil \sixteenpsix %1
        \or \parshape=194 \twollineil \sixteenpsix %2
        \or \parshape=195 \threellineil \sixteenpsix %3
        \or \parshape=196 \fourllineil \sixteenpsix %4
        \or \parshape=197 \fivellineil \sixteenpsix %5
        \or \parshape=198 \sixllineil \sixteenpsix %6
        \or \parshape=199 \slineil\sixllineil \sixteenpsix %1+6
        \or \parshape=200 \twoslineil\sixllineil \sixteenpsix %2+6
        \or \parshape=201 \threeslineil\sixllineil \sixteenpsix %3+6
        \or \parshape=202 \fourslineil\sixllineil \sixteenpsix %4+6
        \or \parshape=203 \fiveslineil\sixllineil \sixteenpsix \fi %5+6
    \fi\fi} %
\newcount\prevlinesremained \prevlinesremained=0 %上一段所剩最後一頁中行
\newcount\ti \newcount\tj % 臨時變量
\def\setnewparshape#1{ %根據上一段剩餘的行數設置新一段的parshape
    \tj=#1 \multiply\tj by 2 %
    \ti=\prevgraf \advance\ti by \prevlinesremained %加上上一段剩下的行
    %\showthe\ti \showthe\tj %
    \modulo{\the\ti}{\the\tj} \prevlinesremained=\tmpmod %
    \advance\tj by -\prevlinesremained %
    %\showthe\tj \showthe\prevlinesremained %
    \setrestedparshape{#1}{\tj}}
%簡單的單層方框
\def\titlebox#1{\hbox{\vbox{
\hrule height1pt\hbox{\vrule width1pt\kern4pc\vrule width1pt
\vbox{\kern9pt#1\kern9pt}\vrule width1pt\kern4pc\vrule width1pt}\hrule height1pt}
}}
%自動加在奇數頁首的雲頭符
\newbox\pghdr%
\TibHeaderBox=\pghdr%
\def\header{{\SamChe ༄༅\kern.01pt།\hskip2.7em plus .5em minus.5em།}}%
\setbox\pghdr=\hbox{\header}%
\Sam
