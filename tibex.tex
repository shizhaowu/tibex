
%
% The Tibetan font definition & glue insertion between
%       letters and punctuations
%
\def\TibeX{\hbox{Tib\kern-.1em \lower.5ex\hbox{E}\kern-.125em X}}
\catcode`\@=11  % 内部標識符的命名中使用@
\catcode`\_=11  % _一般是數學公式下標,藏文排版中將其用於標識符
%
% Sambhota font family
\font\SamThug='SambhotaDege:script=tibt;+abvs;+blws' at 42pt
\font\SamThugu='SambhotaDege:script=tibt;+abvs;+blws' at 36pt
\font\SamkCig='SambhotaDege:script=tibt;+abvs;+blws' at 26pt
\font\SamkCigChung='SambhotaDege:script=tibt;+abvs;+blws' at 24pt
\font\SamkNyis='SambhotaDege:script=tibt;+abvs;+blws' at 22pt
\font\SamkNyisChung='SambhotaDege:script=tibt;+abvs;+blws' at 20pt
\font\SamkSum='SambhotaDege:script=tibt;+abvs;+blws' at 18pt
\font\SamkSumChung='SambhotaDege:script=tibt;+abvs;+blws' at 16pt
\font\SambZhi='SambhotaDege:script=tibt;+abvs;+blws' at 14pt
\font\SambZhiChung='SambhotaDege:script=tibt;+abvs;+blws' at 12pt
\font\SamlNga='SambhotaDege:script=tibt;+abvs;+blws' at 10.5pt
\font\SamlNgaChung='SambhotaDege:script=tibt;+abvs;+blws' at 9pt
\let\Sam=\SamkNyisChung \let\sam=\Sam
\let\SamChe=\SamkNyis
\let\SamAu=\SamkSum
%
% Jomolhari font family
\font\JoThig='Jomolhari:script=tibt;+abvs;+blws' at 42pt
\font\JoThigu='Jomolhari:script=tibt;+abvs;+blws' at 36pt
\font\JokCig='Jomolhari:script=tibt;+abvs;+blws' at 26pt
\font\JokCigChung='Jomolhari:script=tibt;+abvs;+blws' at 24pt
\font\JokNyis='Jomolhari:script=tibt;+abvs;+blws' at 22pt
\font\JokNyisChung='Jomolhari:script=tibt;+abvs;+blws' at 18pt
\font\JokSum='Jomolhari:script=tibt;+abvs;+blws' at 16pt
\font\JokSumChung='Jomolhari:script=tibt;+abvs;+blws' at 15pt
\font\JobZhi='Jomolhari:script=tibt;+abvs;+blws' at 14pt
\font\JobZhiChung='Jomolhari:script=tibt;+abvs;+blws' at 12pt
\font\JolNga='Jomolhari:script=tibt;+abvs;+blws' at 10.5pt
\font\JolNgaChung='Jomolhari:script=tibt;+abvs;+blws' at 9pt
\let\Jo=\JokNyisChung \let\jo=\JokSum
\let\JoChe=\JokNyis
\let\JoAu=\JokSum
%
% 藏研乌坚体 font family
\font\ZhibThig='藏研乌坚体:script=tibt;+abvs;+blws' at 42pt
\font\ZhibThigu='藏研乌坚体:script=tibt;+abvs;+blws' at 36pt
\font\ZhibkCig='藏研乌坚体:script=tibt;+abvs;+blws' at 26pt
\font\ZhibkCigChung='藏研乌坚体:script=tibt;+abvs;+blws' at 24pt
\font\ZhibkNyis='藏研乌坚体:script=tibt;+abvs;+blws' at 22pt
\font\ZhibkNyisChung='藏研乌坚体:script=tibt;+abvs;+blws' at 18pt
\font\ZhibkSum='藏研乌坚体:script=tibt;+abvs;+blws' at 16pt
\font\ZhibkSumChung='藏研乌坚体:script=tibt;+abvs;+blws' at 15pt
\font\ZhibbZhi='藏研乌坚体:script=tibt;+abvs;+blws' at 14pt
\font\ZhibbZhiChung='藏研乌坚体:script=tibt;+abvs;+blws' at 12pt
\font\ZhiblNga='藏研乌坚体:script=tibt;+abvs;+blws' at 10.5pt
\font\ZhiblNgaChung='藏研乌坚体:script=tibt;+abvs;+blws' at 9pt
\let\Zhib=\ZhibkSum \let\zhib=\ZhibkSum
\let\ZhibChe=\ZhibkNyis \let\ZhibAu=\ZhibkSumChung
%
\font\bigtenrm=cmr10 scaled\magstep2
\def\romanfittib#1{\bws\lower9pt\hbox{\bigtenrm #1}\bws}
% glues to be inserted between Tibetan letters & punctuation
\def\LetterSkip{\nobreak\hskip0em plus.015em minus.003em}
\def\WordSkip{\hskip0em plus.05em minus.05em}
\def\BackwordSkip{\hskip.07em plus.03em minus.03em}
\def\SentenceSkip{\hskip.6em plus.3em minus.3em}
\let\ss=\SentenceSkip
\let\ws=\WordSkip
\let\bws=\BackwordSkip
\let\nb=\nobreak
\let\ls=\LetterSkip

% clear XeTeX pre-defined toks
  \XeTeXinterchartoks 0 1 = {}  
  \XeTeXinterchartoks 0 2 = {}  
  \XeTeXinterchartoks 0 3 = {}  
  \XeTeXinterchartoks 1 0 = {}  
  \XeTeXinterchartoks 2 0 = {}  
  \XeTeXinterchartoks 3 0 = {}  
  \XeTeXinterchartoks 1 1 = {}  
  \XeTeXinterchartoks 1 2 = {}  
  \XeTeXinterchartoks 1 3 = {}  
  \XeTeXinterchartoks 2 1 = {}  
  \XeTeXinterchartoks 2 2 = {}
  \XeTeXinterchartoks 2 3 = {}
  \XeTeXinterchartoks 3 1 = {}
  \XeTeXinterchartoks 3 2 = {}
  \XeTeXinterchartoks 3 3 = {}

\newcount\c@n
% 一个藏文字符位置上的最上面的部分是 class 4
\c@n="F40 \loop \ifnum\c@n<"F6D \global\XeTeXcharclass\c@n=4 \advance\c@n by 1\repeat
% 元音以及梵文的长音符号是 class 5
\c@n="F91 \loop \ifnum\c@n<"FBC \global\XeTeXcharclass\c@n=5 \advance\c@n by 1\repeat
% 一个藏文字符位置上的其他的部分是 class 6
\c@n="F71 \loop \ifnum\c@n<"F82 \global\XeTeXcharclass\c@n=6 \advance\c@n by 1\repeat
% 有后面的长脚的字母是 class 7
\XeTeXcharclass `\ཀ = 7
\XeTeXcharclass `\ག = 7
\XeTeXcharclass `\ཤ = 7
%
\XeTeXinterchartoks 4 4 = {\LetterSkip}
\XeTeXinterchartoks 4 7 = {\LetterSkip}
\XeTeXinterchartoks 7 4 = {\LetterSkip}
\XeTeXinterchartoks 7 7 = {\LetterSkip}
%
\XeTeXinterchartoks 5 4 = {\LetterSkip}
\XeTeXinterchartoks 5 7 = {\LetterSkip}
\XeTeXinterchartoks 6 4 = {\LetterSkip}
\XeTeXinterchartoks 6 7 = {\LetterSkip}
%
% 分隔两个音节的标点་ 是 class 12
\XeTeXcharclass `\་ = 12
\XeTeXinterchartoks 12 4 = {\WordSkip}
\XeTeXinterchartoks 12 7 = {\WordSkip}
\XeTeXinterchartoks 4 12 = {\nobreak\WordSkip}
\XeTeXinterchartoks 7 12 = {\nobreak\WordSkip}
\XeTeXinterchartoks 5 12 = {\nobreak\WordSkip}
\XeTeXinterchartoks 6 12 = {\nobreak\WordSkip}
%
% 表示一句结束的标点། 是 class 11
\XeTeXcharclass `\། = 11
% 藏文中出现在།之后的空格、回车和制表符需要作特殊处理,因为这是一句的结束
% 用\XeTeXinterchartoks 来定义不知何故无用,因此改用sed来替换添加glue
%\XeTeXcharclass `\  = 10
%\XeTeXcharclass `\^^J = 10
%\XeTeXcharclass `\^^I = 10
%
%\XeTeXinterchartoks 11 10 = {\SentenceSkip\message{Space after shad}}
%\XeTeXinterchartoks 7 10 = {\SentenceSkip\message{Space after backfoot}}
%
\XeTeXinterchartoks 4 11 = {\nobreak\WordSkip}
\XeTeXinterchartoks 5 11 = {\nobreak\WordSkip}
\XeTeXinterchartoks 6 11 = {\nobreak\WordSkip}
\XeTeXinterchartoks 7 11 = {\nobreak\SentenceSkip}
\XeTeXinterchartoks 12 11 = {\nobreak\WordSkip}
\XeTeXinterchartoks 11 11 = {\nobreak\SentenceSkip}
\XeTeXinterchartoks 11 4 = {\BackwordSkip}
\XeTeXinterchartoks 11 7 = {\BackwordSkip}
%
\def\NoInterLetterSkip{
\XeTeXinterchartoks 4 4 = {}
\XeTeXinterchartoks 4 7 = {}
\XeTeXinterchartoks 7 4 = {}
\XeTeXinterchartoks 7 7 = {}
%
\XeTeXinterchartoks 5 4 = {}
\XeTeXinterchartoks 5 7 = {}
\XeTeXinterchartoks 6 4 = {}
\XeTeXinterchartoks 6 7 = {}
}
%
\def\DontSplitDblShad{
\XeTeXinterchartoks 7 11 = {\nobreak\WordSkip}
\XeTeXinterchartoks 11 11 = {\nobreak\WordSkip}
\XeTeXinterchartoks 11 4 = {\SentenceSkip}
\XeTeXinterchartoks 11 7 = {\SentenceSkip}
}
%
\XeTeXinterchartokenstate=1
%
\RinChenShadLetters=2
%
\spaceskip=.07em
\xspaceskip=.07em
%
% 把一个小于等于999的数转换成藏文
%
\newif\ifFifteenWaZur \FifteenWaZurtrue % 控制15转换为བཅྭོ་ལྔ,若爲false則爲བཅོ་ལྔ
\newcount\c@n \newcount\c@m \newcount\c@t %
%\newcount\centi \newcount\tens %
\newif\ifspeci@ltens
\def\tibnumber#1{
\c@n=#1
% calculate the centi digit
\c@t=\c@n \divide\c@t by 100
   \c@m=\c@t \multiply\c@m by 100
   \advance\c@n by -\c@m
% now t contains the centi digit, n contains the lower two
\ifnum\c@n=0 % if the lower two digits is 00, the following Tseg is not needed
  \ifcase\c@t\or བརྒྱ\or ཉིས་བརྒྱ\or སུམ་བརྒྱ\or བཞི་བརྒྱ\or ལྔ་བརྒྱ\or དྲུག་བརྒྱ
            \or བདུན་བརྒྱ\or བརྒྱད་བརྒྱ\or དགུ་བརྒྱ\fi\relax
\else
  \ifcase\c@t\or བརྒྱ་\or ཉིས་བརྒྱ་\or སུམ་བརྒྱ་\or བཞི་བརྒྱ་\or ལྔ་བརྒྱ་\or དྲུག་བརྒྱ་
      \or བདུན་བརྒྱ\or བརྒྱད་བརྒྱ་\or དགུ་བརྒྱ་\fi\relax
  % calculate the tens digit
  \c@t=\c@n \divide\c@t by 10
    \c@m=\c@t \multiply\c@m by 10
    \advance\c@n by -\c@m
  % now t contains the tens digit
  \ifnum\c@n=0 % if the last digit is 0, the following Tseg is not needed either
    \ifcase\c@t\or བཅུ\or ཉི་ཤུ\or སུམ་ཅུ\or བཞི་བཅུ\or ལྔ་བཅུ\or དྲུག་ཅུ\or བདུན་ཅུ
       \or བརྒྱད་ཅུ\or དགུ་བཅུ\fi\relax %
  \else %
    \ifnum\c@t=1 %
        \speci@ltensfalse %
        \ifnum\c@n=5 \speci@ltenstrue\fi %
        \ifnum\c@n=8 \speci@ltenstrue\fi %
        \ifspeci@ltens %
            \ifFifteenWaZur བཅྭོ་ \else བཅོ་ \fi %
        \else བཅུ་\fi \fi %
    \ifcase\c@t\or \or ཉེར་\or སོ་\or ཞེ་\or ང་\or རེ་\or དོན་\or གྱ་\or གོ་\fi\relax
    \ifcase\c@n\or གཅིག\or གཉིས\or གསུམ\or བཞི\or ལྔ\or དྲུག\or བདུན\or བརྒྱད\or དགུ \fi\relax
  \fi
\fi
}

%
% 將一个小于等于999的数轉換爲藏文༠-༩的數字表示
%
\def\tibdigit#1{
%        \message{\the#1}
	\ifcase#1 ༠ \or ༡ \or ༢ \or ༣ \or ༤ \or ༥ \or ༦ \or ༧ \or ༨ \or ༩ \fi\relax
}
\newif\ifFirstDigit
\def\tibfolio#1{
%\tracingcommands=2
\FirstDigittrue
\c@n=#1
% calculate the centi digit
\c@t=\c@n \divide\c@t by 100
   \c@m=\c@t \multiply\c@m by 100
   \advance\c@n by -\c@m
% now t contains the centi digit, n contains the lower two
\ifnum\c@t=0
	\relax
\else
	\tibdigit{\c@t}
	\FirstDigitFalse
\fi
% calculate the tens digit
\c@t=\c@n \divide\c@t by 10
\c@m=\c@t \multiply\c@m by 10
\advance\c@n by -\c@m
% now t contains the tens digit
\ifFirstDigit
	\ifnum\c@t=0
		\relax
	\else
		\tibdigit\c@t
	\fi
\else
	\tibdigit\c@t
\fi
% n contains the last digit
\tibdigit\c@n
%\tracingcommands=0
}
